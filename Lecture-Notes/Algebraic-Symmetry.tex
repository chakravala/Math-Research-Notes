\documentclass[]{article}
\usepackage[latin1]{inputenc}
\usepackage{graphicx}
\usepackage[left=1.00in, right=1.00in, top=1.10in, bottom=1.00in]{geometry}

\usepackage{dirtytalk}
\usepackage[normalem]{ulem}
\usepackage{tikz-cd}
\usepackage{units}
\usepackage{algorithm}
\usepackage{algpseudocode}
\usepackage{alltt}
\usepackage{mathrsfs}
\usepackage{amssymb}
\usepackage{amsmath}
\DeclareMathOperator\cis{cis}

% (font shortcuts)
\usepackage{amsfonts}
\newcommand{\mb}[1]{\mathbb{#1}}
\newcommand{\mc}[1]{\mathcal{#1}}
\newcommand{\ms}[1]{\mathscr{#1}}
\newcommand{\mf}[1]{\frak{#1}}

% (arrow shortcuts)
\newcommand{\ra}{\rightarrow}
\newcommand{\lra}{\longrightarrow}
\newcommand{\la}{\leftarrow}
\newcommand{\lla}{\longleftarrow}
\newcommand{\Ra}{\Rightarrow}
\newcommand{\Lra}{\Longrightarrow}
\newcommand{\La}{\Leftarrow}
\newcommand{\Lla}{\Longleftarrow}
\newcommand{\lr}{\leftrightarrow}
\newcommand{\llr}{\longleftrightarrow}
\newcommand{\Lr}{\Leftrightarrow}
\newcommand{\Llr}{\Longleftrightarrow}

% (match parenthesis)
\newcommand{\mlr}[1]{\left|#1\right|}
\newcommand{\plr}[1]{\left(#1\right)}
\newcommand{\blr}[1]{\left[#1\right]}

% (exponent shortcuts)
\newcommand{\inv}{^{-1}}
\newcommand{\nrt}[2]{\sqrt[\leftroot{-2}\uproot{2}#1]{#2}}

% (annotation shortcuts)
\newcommand{\conj}[1]{\overline{#1}}
\newcommand{\ol}[1]{\overline{#1}}
\newcommand{\ul}[1]{\underline{#1}}
\newcommand{\os}[2]{\overset{#1}{#2}}
\newcommand{\us}[2]{\underset{#1}{#2}}
\newcommand{\ob}[2]{\overbrace{#2}^{#1}}
\newcommand{\ub}[2]{\underbrace{#2}_{#1}}
\newcommand{\bs}{\backslash}
\newcommand{\ds}{\displaystyle}

% (set builder)
\newcommand{\set}[1]{\left\{ #1 \right\}}
\newcommand{\setc}[2]{\left\{ #1 : #2 \right\}}
\newcommand{\setm}[2]{\left\{ #1 \, \middle| \, #2 \right\}}

% (group generator)
\newcommand{\gen}[1]{\langle #1 \rangle}

% (functions)
\newcommand{\im}[1]{\text{im}(#1)}
\newcommand{\range}[1]{\text{range}(#1)}
\newcommand{\domain}[1]{\text{domain}(#1)}
\newcommand{\dist}[1]{(#1)}
\newcommand{\sgn}{\text{sgn}}

% (Linear Algebra)
\newcommand{\mat}[1]{\begin{bmatrix}#1\end{bmatrix}}
\newcommand{\pmat}[1]{\begin{pmatrix}#1\end{pmatrix}}
%\newcommand{\dim}[1]{\text{dim}(#1)}
\newcommand{\rnk}[1]{\text{rank}(#1)}
\newcommand{\nul}[1]{\text{nul}(#1)}
\newcommand{\spn}[1]{\text{span}\,#1}
\newcommand{\col}[1]{\text{col}(#1)}
%\newcommand{\ker}[1]{\text{ker}(#1)}
\newcommand{\row}[1]{\text{row}(#1)}
\newcommand{\area}[1]{\text{area}(#1)}
\newcommand{\nullity}[1]{\text{nullity}(#1)}
\newcommand{\proj}[2]{\text{proj}_{#1}\left(#2\right)}
\newcommand{\diam}[1]{\text{diam}\,#1}

% (Vectors common)
\newcommand{\myvec}[1]{\vec{#1}}
\newcommand{\va}{\myvec{a}}
\newcommand{\vb}{\myvec{b}}
\newcommand{\vc}{\myvec{c}}
\newcommand{\vd}{\myvec{d}}
\newcommand{\ve}{\myvec{e}}
\newcommand{\vf}{\myvec{f}}
\newcommand{\vg}{\myvec{g}}
\newcommand{\vh}{\myvec{h}}
\newcommand{\vi}{\myvec{i}}
\newcommand{\vj}{\myvec{j}}
\newcommand{\vk}{\myvec{k}}
\newcommand{\vl}{\myvec{l}}
\newcommand{\vm}{\myvec{m}}
\newcommand{\vn}{\myvec{n}}
\newcommand{\vo}{\myvec{o}}
\newcommand{\vp}{\myvec{p}}
\newcommand{\vq}{\myvec{q}}
\newcommand{\vr}{\myvec{r}}
\newcommand{\vs}{\myvec{s}}
\newcommand{\vt}{\myvec{t}}
\newcommand{\vu}{\myvec{u}}
\newcommand{\vv}{\myvec{v}}
\newcommand{\vw}{\myvec{w}}
\newcommand{\vx}{\myvec{x}}
\newcommand{\vy}{\myvec{y}}
\newcommand{\vz}{\myvec{z}}
\newcommand{\vzero}{\myvec{0}}

% Theorems and Propositions
\usepackage{amsthm}
\newtheorem{theorem}{Theorem}
\newtheorem{proposition}{Proposition}

\theoremstyle{definition}
\newtheorem{definition}{Definition}

\theoremstyle{remark}
\newtheorem*{remark}{Remark}
\newtheorem{example}{Example}
\newtheorem*{recall}{Recall}
\newtheorem*{note}{Note}
\newtheorem*{observe}{Observe}
\newtheorem*{question}{\underline{Question}}
\newtheorem*{fact}{Fact}
\newtheorem{corollary}{Corollary}
\newtheorem*{lemma}{Lemma}
\newtheorem{xca}{Exercise}

%\usepackage[active,tightpage]{preview}
\setlength\PreviewBorder{7.77pt}
\usepackage{varwidth}
\AtBeginDocument{\begin{preview}\begin{varwidth}{\linewidth}}
\AtEndDocument{\end{varwidth}\end{preview}}


\author{Presenter: Dan Yasaki, Notes by Michael Reed, Book: Gallian}
\title{Abstract Algebra}
%date{}

\begin{document}
\maketitle

%\begin{abstract}
%\end{abstract}

\ul{Groups}, Rings, Fields

\section{Introduction to Rings}

\begin{definition}
	A \ul{ring} is a set $R$ with two binary operations, addition \say{$a+b$} and multiplication \say{$ab$} such that
	\begin{itemize}
		\item[(i)] $a+b=b+a$, for all $a,b\in R$.
		\item[(ii)] $(a+b)+c=a+(b+c)$ for all $a,b,c\in R$.
		\item[(iii)] There exists an element $\in R$ such that $0+a=a+0$ for all $a\in R$.
		\item[(iv)] For each $a\in R$, there exists $-a\in R$ such that $a+(-a)=0$.
	\end{itemize}
	(i)-(iv) imply $(R,+)$ is an Abelian group.
	\begin{itemize}
		\item[(v)] $a(bc) = (ab)c$ for all $a,b,c\in R$.
		\item[(vi)] $a(b+c)=ab+ac$ for all $a,b,c\in R$.
	\end{itemize}
\end{definition}
\begin{remark}
	A ring may not have a multiplicative identity.
	\begin{itemize}
		\item When it does, we call it a \ul{unity}, denoted often as 1.
			[if a ring has unity, then it is unique.] (exercise)
		\item elements of a ring need not have multiplicative inverses.
			If $a\in R$ has a multiplicative inverse, we denote it $a\inv$ and call $a$ a \ul{unit}.
		\item we use words like \say{factor} or \say{divides} since rings have multiplication. e.g., if $a,b\in R$ and $ab=c$, then $a$ is a factor of $c$ or $a$ divides $c$.
		\item For $n\in\mb Z_{>0}$, $a\in R$, define $n\cdot a = a+a+\dots+a$.
	\end{itemize}
\end{remark}

\begin{example}
	\begin{itemize}
		\item[(i)] $\mb Z$ with regular addition and regular multiplication is a commutative ring with unity 1.
			The units are $\set{1,-1} = U(\mb Z) = \mb Z^\times = \mb Z^*$.i
		\item[(ii)] Fix positive integer $n$, $\mb Z_n=\set{0,1,2,\dots,n-1}$ with addition modulo $n$ and multiplication modulo $n$ is a commutative ring with unity.
			The units are $\mb Z_n^\times U(n) = \setm{a\in\mb Z_n}{\gcd(a,n)=1}$.
		\item[(iii)] $2\mb Z=\setm{2k}{k\in\mb Z}$ with regular addition and multiplication is a commutative ring without unity.
		\item[(iv)] Let $M_2(\mb R)=\text{Mat}_{2\times2}(\mb R) = \setm{\mat{a&b\\c&d}}{a,b,c,d\in\mb R}$ with matrix addition and multiplication.
			This is a non-commutative ring with unity $I=\mat{1&0\\0&1}$.
		\item[(v)] $C^\infty(\mb R) = \setm{f:\mb R\ra\mb R}{f\text{ is infinitely differentiable}}$ with addition given by: for $f,g\in C^\infty(\mb R)$, $(f+g)(x) = f(x)+g(x)$ and multiplication given by  $(fg)(x) = f(x)g(x)$.
			Additive identity is $z(x)=0$, unity $u(x)=1$. $C^\infty(\mb R)$ is a commutative ring with unity $u(x)=1$.
	\end{itemize}
\end{example}

\ul{Notation}: for $a,b\in R$, $a-b$ is $a+(-b)$.
\ul{\say{Rules} for multiplication}: Let $R$ be a ring. Let $a,b\in R$.
\begin{itemize}
	\item[(i)] $0a=a0=0$
	\item[(ii)] $a(-b)=(-a)b = -(ab)$
	\item[(iii)] $(-a)(-b) = ab$
	\item[(iv)] $a(b-c)=ab-ac$, $(b-c)a=ba-ca$
\end{itemize}
Furthermore, if $R$ has unity 1,
\begin{itemize}
	\item[(v)] $(-1)a=-a$
	\item[(vi)] $(-1)(-1)=1$
\end{itemize}
\begin{proof}
	Exercise.
\end{proof}

\begin{definition}
	A subset $S$ of a ring $R$ is a \ul{subring} if $S$ is a ring with the operations of $\mb R$.
\end{definition}
\ul{Subring Test}: A subset $S$ of a ring $R$ is a subring if
\begin{itemize}
	\item[(i)] $S=\emptyset$
	\item[(ii)] for every $a,b\in S$, $a-b\in S$.
	\item[(iii)] for every $a,b\in S$, $ab\in S$.
\end{itemize}
\begin{example}
	Let $R$ be a ring. Then $R$ is a subring of $R$. $\set{0_R}$ is a subring of $R$.
\end{example}
\begin{example}
	$S = \set{0,2,4}\subseteq\mb Z_6$ then $S$ is a subring of $\mb Z_6$.
\end{example}
\begin{example}
	Let $S=\setm{\mat{a&0\\0&b}}{a,b\in\mb Z}\subseteq M_2(\mb Z)$.
\end{example}
\begin{proof}
	\begin{itemize}
		\item[(i)] $\mat{1&0\\0&1}\in S$ so $S\neq\emptyset$.
		\item[(ii)] Let $A=\mat{a_1&0\\0&a_1}$ and $B=\mat{b_1&0\\0&b_2}$ be elements of $S$. $A-B=\mat{a_1-b_1&0\\0&a_2b-2} \in S$ [since $a_1-b_2\in\mb Z$ and $a_2-b_2\in\mb Z$ and off-diag are 0 $A-B\in S$.]
		\item[(iii)] Let $A=\mat{a_1b_1&0\\0&a_2b_2}\in S$.
	\end{itemize}
	Thus by Subring Test, $S$ is a subring of $M_2(\mb Z)$.
\end{proof}

\section{Integral Domains}
\begin{definition}
	A \ul{zero divisor} in a commutative ring $R$ is a non-zero element $a\in R$ such that there exists a non-zero element $b\in R$ with $ab=0$.
\end{definition}
\begin{definition}
	An \ul{integral domain} is a commutative ring with unity with no zero-divisors.
\end{definition}
\begin{remark}
	If $R$ is an integral domain and $a,b\in R$ with $ab=0$, then $a=0$ or $b=0$.
\end{remark}
\begin{example}
	$\mb Z$ is an integral domain.
\end{example}
\begin{example}
	$\mb Z_6$ is not an integral domain since $\alpha\neq0$ and $3\neq0$ but \say{$\set{0,1,2,3,4,5}$}, $2\cdot3=0$ is $\mb Z_6$.
\end{example}
\begin{example}
	$\mb Z_5$ \ul{is} an integral domain. (exercise).
\end{example}
More generally $\mb Z_p$ is an integral domain for any prime $p$ (exercise) [Hint: for $a,b\in\mb Z$ if $p\mid(ab)$ and $p$ prime, then $p\mid a$ or $p\mid b$.]

Last time: \ul{\ul{Roughly}}, a ring $R$ is an abelian group $(R,+)$ with an associative binary operation (multiplication) on $R$ such that multiplication and addition play nice.
An integral domain is a commutative ring with unity and no zero-divisors.

\begin{theorem}
	[Cancellation Theorem for integral domains]
	Let $R$ be an integral domain. Let $a,b,c\in R$ with $a\neq0$ and $ab=ac$. Then $b=c$.
\end{theorem}
\begin{proof}
	Suppose $ab=ac$ and $a\neq0$. $ab-ac=0$, so $a(b-c)=0$.
	Since $R$ is an integral domain and $a\neq0$, $b-c=0$. Thus $b=c$.
\end{proof}
\begin{definition}
	A \ul{field} is a commutative ring with unity in which every nonzero element is a unit.
\end{definition}
\begin{example}
	$\mb Q,\mb C,\mb R,\mb Z_p$ for $p$ prime.
\end{example}
\begin{remark}
	A field is an integral domain.
\end{remark}
\begin{proof}
	Let $F$ be a field. Let $a,b\in F$ with $a\neq0$ and $ab=0$.
	We just need to show $b=0$. Then $a\neq0$ implies $a\inv$ exists. So $a\inv(ab)=a\inv\cdot 0$ and $(a\inv a)b=0$ imply $b=0$.
	So $F$ has no zero-divisors, hence $F$ is an integral domain.
\end{proof}
\begin{theorem}
	Let $R$ be a finite integral domain. Then $R$ is a field.
\end{theorem}
\begin{proof}
	We just need to show every non-zero element in $R$ is a unit.
	Let $a\in R,a\neq0$.
	\ul{Case 1}: $a=1$. $a\inv=1\in R$.
	\ul{Case 2}: $a\neq1$. Consider $a,a^2,a^3,a^4,\dots,a^i,\dots,a^j,d\dots$.
	There exists $i,j$ positive integers such that $a^i=a^j$ and $i\neq j$.
	Without loss of generality, assume $i<j$. Then $a^i=a^j$ implies $1=a^{j-i}$ by cancellation theorem.
	Then $1=a^{j-i-1}a$.
	\begin{note}
		since $i<j$, $j-i-1\geq0$. So $a^{j-i-1}$ is $a\inv$, so $a$ is a unit.
	\end{note}
\end{proof}
\begin{corollary}
	Let $p$ be prime. Then $\mb Z_p$ is a field.
\end{corollary}
\begin{proof}
	$\mb Z_p$ is finite, so it suffices to show $\mb Z_p$ is an integral domain.
	It is clear that $\mb Z_p$ is a commutative ring with unity.
	Let $a,b\in\mb Z_p$ with $a\neq0$ and $ab=0$ in $\mb Z_p$.
	We need to show $b=0$ in $\mb Z_p$. Since $ab=0$ in $\mb Z_p$, there exists a $k\in\mb Z$ such that $ab=p$.
	$p$ is prime, so $p\mid pk$, so $p\mid ab$, so $p\mid a$ or $p\mid b$.
	Since $a\neq0$ in $\mb Z_p$, $p\nmid a$ so $p\mid b$. Thus $b=0$ in $\mb Z_p$.
\end{proof}
\begin{example}
	Let $\mb Z_3[i]=\setm{a+ib}{a,b\in\mb Z_3,i^2}$ where addition and multiplication is \say{nice} and $i^2=-1$.
	\begin{align*}
		(a+ib)+(a'+ib') &= (a+a')+i(b+b') \\
		(a+ib)(a'+ib') &= aa'+iab'+iba'+\ub{-1}{i^2}bb' = (aa'-bb')+i(ab'+ba')
	\end{align*}
	\ul{Claim}: $\mb Z_3[i]$ is a field.
	Show $\mb Z_3[i]$ is an integral domain by showing it has no zero-divisors. Write multiplication table (exercise).
\end{example}
\begin{example}
	$\mb Q[\sqrt2] = \setm{a+\sqrt2b}{a,b\in\mb Q}\subseteq\mb R$.
	Then $\mb Q[\sqrt2]$ is a field.
	Let $a+\sqrt2b\in\mb Q[\sqrt2]\bs\set0$. What is $(a+\sqrt2b)\inv$?
	$$ \frac1{a+\sqrt2b}\cdot\plr{\frac{a-\sqrt2b}{a-\sqrt2b}} = \frac{a-\sqrt2b}{a^2-2b} = \us{\in\mb Q}{\frac a{a^2-2b}} + \us{\in\mb Q}{\plr{\frac{-1}{a^2-2b}}}\sqrt2 $$
\end{example}
\begin{definition}
	Let $R$ be a ring. The \ul{characteristic of $R$}, denoted $\text{char}(R)$, is the least positive integer $n$ such that $\ub{=\ub{n\text{ times}}{x+x+\dots+x}}{n\cdot x=0}$ for all $x\in R$.
	If no such integer exists, we say the characteristic of $R$ is 0.
\end{definition}
\begin{example}
	$\text{char}(\mb Z_4)=4$, $\text{char}(\mb Z_n)=n$, $\text{char}(\mb Z)=0$
\end{example}
\begin{theorem}
	\label{thm-*}
	Let $R$ be a ring with unity 1.
	\begin{enumerate}
		\item if 1 has infinite order in $(R,+)$, then $\text{char}(R)=0$.
		\item if 1 has finite order $n$ in $(R,+)$, then $\text{char}(R)=n$.
	\end{enumerate}
\end{theorem}
\begin{proof}
	exercise.
\end{proof}
\begin{theorem}
	Let $R$ be an integral domain. Then $\text{char}(R)=0$ or $\text{char}(R)=p$, $p$ prime.
\end{theorem}
\begin{proof}
	\ul{Case 1}: The order of 1 in $(R,+)$ is infinite. Then $\text{char}(R)=0$ by Theorem \ref{thm-*}. Done.
	\ul{Case 2}: Suppose the order of 1 in $(R,+)$ is finite integer $n$
	Suppose $n=s\cdot t$, $s,t\in\mb Z$. $1\leq s,t\leq n$.
	Then $(st)\cdot1=(s\cdot1)(t\cdot 1)=0$. $R$ is an integral domain, so $s\cdot1=0$ or $t\cdot1=0$.
	Thus $s=n$ and $t=1$ or $s=1$ and $t=n$. Thus $n$ is prime.
\end{proof}
Suppose $R$ is a commutative ring with unity. For $n\in\mb Z_{>0}$, view $n$ as element of $R$, $$n=n\cdot 1=\ub{n\text{ times}}{1+1+\dots+1}.$$

Consider $x^2-4x+3=0$ or $(x-3)(x-1)=0$.
If $R$ is an integral domain, then $x-3=0$ or $x-1=0$, so $x=3$ or $x=1$.
\begin{example}
	Let $R=\mb Z_{12}$. Brute force check by plugging in elements of $\mb Z_{12}$ into $x^2-4x+3$, to see when 0. $x=3,x=1$ work.
	\ul{Claim}: $x=7$ and $x=9$ also work.
\end{example}

\section{Ideals and Factor Rings (Quotient Rings)}
Let $G$ be a group. $N\leq G$. $G/N$ is a group precisely when $N$ is normal subgroup of $G$.
\begin{definition}
	A subring $A$ of a ring $R$ is a \ul{(two-sided) ideal} of $R$ if for every $r\in R$ and every $a\in A$, $ar$ and $ra$ are in $A$.
\end{definition}
\begin{remark}
	\begin{enumerate}
		\item if $R$ is commutative, $ar=ra$ so only check one.
		\item all ideals in our course are two-sided ideals, so \say{two-sided} is dropped from name.
	\end{enumerate}
\end{remark}
\begin{example}
	Let $R=\mb Z$. Let $A=\setm{5k}{k\in\mb Z}$. $A$ is a subring of $\mb Z$ by subring test.
	Let $r\in\mb Z$, let $a\in A$. Then there exists $k\in\mb Z$ such that $a=5k$. $ra=ar=5(kr)\in A$ since $kr\in\mb Z$.
	So $A$ is an ideal of $\mb Z$.
	$A$ is an example of a principal ideal. $A=\gen5=5\mb Z$.
\end{example}

\subsection{External Direct Products}

Let $G$ and $H$ be groups. The \ul{external direct product of $G$ and $H$}, denoted by $G\times H$ or $G\oplus H$, is
$$ G\times H = \setm{(g,h)}{g\in G,h\in H}$$
with binary operation: for $(g,h),(g',h')\in G\times H$ the product is $$(g,h)(g',h') = (gg',hh').$$
\begin{xca}
	$G\times H$ is a group.
\end{xca}
More generally, for groups $G_1,G_2,\dots,G_n$ the direct product, denoted $G_1\times G_2\times\cdots\times G_n$ is $G_1\times G_2\times\cdots\times G_n = \setm{(g_1,g_2,\dots,g_n)}{g_i\in G_i}$ with componentwise group operation.

\begin{example}
	$G = \mb Z_2\times\mb Z_3 = \set{(0,0),(0,1),(0,2),(1,0),(1,1),(1,2)}$
	Then $$(1,0)+_G(1,1) = (1+_{\mb Z_2}1,0+_{\mb Z_3}1) = (0,1) $$
	$$ (1,1)+(1,1) = (0,2) $$
	$$ (0,1)+(1,1) = (1,0) $$
\end{example}
\begin{note}
	$G$ is cyclic, generated by $(1,1)$, so $G\cong\mb Z_6\cong\mb Z_2\times\mb Z_3$.
\end{note}
\begin{fact}
	[Useful fact] Let $G,H$ be finite cyclic groups.
	Then $G\times H$ is cyclic if and only if $|G|$ and $|H|$ are relatively prime.
\end{fact}
\begin{corollary}
	Let $m,n$ be positive integers. $\mb Z_{mn}\cong\mb Z_m\times\mb Z_n$ if and only if $\gcd(m,n)=1$.
\end{corollary}
\begin{example}
	$\mb Z_{12}\cong\mb Z_4\times\mb Z_3$, but not $\mb Z_2\times\mb Z_6 = \mb Z_2\times\mb Z_2\times\mb Z_3$.
\end{example}

\subsection{Fundamental Theorem of finite abelian groups}

\begin{theorem}
	Every finite abelian group is a direct product of cyclic groups of prime power order.
	Moreover, the number of terms in the decomposition and the orders of the cyclic groups are uniquely determined by the group.
\end{theorem}
\begin{example}
	$\mb Z_{10}\cong\mb Z_2\times\mb Z_5$.
\end{example}
\begin{example}
	$\mb Z_{24}\cong \mb Z^{2^3}\times\mb Z_3$.
\end{example}
\begin{example}
	$D_{10}$ (order 10 dihedral group)? not abelian!
\end{example}
\begin{question}
	What are the isomorphism classes of abelian groups of order $7^4$.
\end{question}
\ul{Partitions of 4}: $\mb Z_{7^4}$, $\mb Z_{7^2}\times\mb Z_{7^2}$, $\mb Z_7\times\mb Z_{7^3}$, $\mb Z_7\times\mb Z_7\times\mb Z_{7^2}$, $\mb Z_7\times\mb Z_7\times\mb Z_7\times\mb Z_7$.
\begin{question}
	isomorphism classes for abelian grous of order $3^2\times 7^4$?
\end{question}
\ul{Partitions of 2}: $\mb Z_{3^2}$ and $\mb Z_3\times\mb Z_2$.
There are $5\times2=10$ different abelian groups of order $3^2\times7^4$.

\section{Ideals and Factor Rings (\ul{\ul{Quotient}} rings)}

\begin{definition}
	A subring $A$ of a ring $R$ is a \ul{(two-sided) ideal of $R$} if for every element $r\in R$ and every $a\in A$, the product $ar$ and $ra$ are both in $A$.
\end{definition}

\begin{theorem}
	[Ideal test] A subset $A\subseteq R$ is an ideal if
	\begin{itemize}
		\item[(i)] $A\neq\emptyset$
		\item[(ii)] $a-b\in A$ for every $a,b\in A$
		\item[(iii)] $ra$ and $ar$ are in $A$ for every $r\in R$ and $a\in A$.
	\end{itemize}
\end{theorem}

\begin{definition}
	Let $R$ be commutative ring with unity. Let $a\in R$.
	The \ul{principal ideal generated by $a$}, denoted $\gen a$, is $\gen a = \setm{ra}{r\in R}$.
	\begin{enumerate}
		\item[$*$] $\gen a$ is an ideal. (exercise)
	\end{enumerate}
\end{definition}

More generally, if $a_1,a_2,\dots,a_n\in R$, the \ul{ideal generated by $a_1,a_2,\dots,a_n$}, denoted $\gen{a_1,a_2,\dots,a_n}$, is $\gen{a_1,a_2,\dots,a_n} = \setm{r_1a_1+r_2a_2+\dots+r_na_n}{r_i\in R}$.
\begin{example}
	$\mb R=\mb Z$, then $\gen{5,7} = \setm{5s+7t}{s,t\in\mb Z} = \gen1=\mb Z$, since Bezout $\exists s,t\in\mb Z$ such that $52+7t=\ub{1}{\gcd(5,7)}$.
\end{example}
In general, for $m,n\in\mb Z$, $\gen{m,n}=\gen{\gcd(m,n)}$.

\begin{example}
	Let $\mb Z[x] = $ ring of polynomials with coefficients in $\mb Z$ in variable $x$.
	Consider the ideal $I=\gen{x,2}$, then $I=\setm{f(x)x+g(x)2}{f(x),g(x)\in\mb Z[x]}$, e.g. contains $x+2,x,x^2,516\in I$, but not $x+3,1\notin I$.

	Let $J = \setm{f(x)\in\mb Z[x]}{f(0)\text{ is even}}$.
	\ul{Claim}: $I=J$.
\end{example}
\begin{proof}
	We show $I\subseteq J$ and $J\subseteq I$.

	\ul{$I\subseteq J$}: let $p(x)\in I$. We want to show $p(x)\in J$.
	Then there exists polynomials $f(x)$ and $g(x)$ in $\mb Z[x]$ such that $p(x)=f(x)\cdot x+g(x)\cdot2$.
	We compute: $p(0)=f(0)\cdot0+g(0)\cdot2 = 2g(0)$ and $g(0)\in\mb Z$, so $p(0)$ is even.
	$\therefore p(x)\in J$ so $I\subseteq J$.

	\ul{$J\subseteq I$}: let $q(x)\in J$. We want to show $g(x)\in I$.
	Then $q(0)$ is even. There exists integers $a_0,a_1,\dots,a_n$ such that $q(x)=a_0+a_1x+\dots+a_nx^n$, so $a_0$ is even.
	There exists integer $k$ such that $a_0=2k$.
	Then $q(x) = 2k+x(a_1+a_2x+\dots+a_nx^{n-1})$, finish at home.
\end{proof} 

Factor Rings (Quotient Ring)

Let $R$ be a ring and let $A\subseteq R$ be an ideal.
$R/A=\setm{r+A}{r\in R}$
\begin{note}
	$r+A = \setm{r+a}{a\in A}$
\end{note}
Endow $R/A$ with add \& multiplication:
\begin{align*}
	(r+A)+(s+A) &= (r+s)+A \\
	(r+A)(s+A) &= (rs)+A
\end{align*}
Show this is well-defined i.e., show this does not depend on the representatives chosen.
Suppose $r,s,r',s'\in R$ such that $r'=r+a$ for some $a\in A$ and $s'=s+b$ for some $b\in A$.
\begin{note}
	$r+A=r'+A$, $s+A=s'+A$.
\end{note}
Let's show $(r'+A)+(s'+A)=(r+A)+(s+A)$. LHS $(r'+s')+A$, RHS $(r+s)+A$.
We need to show $(r'+s')-(r+s)\in A$.
Compute: $r'+s'-r-s = r+a + s+b-r-s = a+b\in A$ since $A$ is an ideal and $a,b\in A$.
Let's show $(r'+A)(s'+A) = (r+A)(s+A)$. LHS: $(r's')+A$, RHS: $(rs)+A$.
We need to show $r's'-rs\in A$. Compute:
$r's'-rs = (r+a)(s+b)-rs = rs+as+ab+rb-rs = as+ab+rb\in A$ since $A$ is an ideal.
$\therefore$ operations are well defined.
\begin{xca}
	show $R/A$ is a ring.
\end{xca}
\begin{example}
	$R=\mb Z$, $A=\gen5=5\mb Z$.
	$R/A = \mb Z/5\mb Z = \setm{r+5\mb Z}{r\in\mb Z}$, $|R/A|=5$, absorb multiples of 5 \say{$5=0$ in the quotient,} \say{$\mb Z/5\mb Z\cong\mb Z_5$.}
\end{example}
\begin{example}
	$R=\mb Z[x]$, $A=\gen{x^2+1} = \setm{f(x)(x^2+1)}{f(x)\in\mb Z[x]}$,
	$$R/A = \mb Z[x]/\gen{x^2+1} = \setm{p(x)+\gen{p(x)+\gen{x^2+1}}}{p(x)\in\mb Z[x]}$$.
	Think of $x^2+1$ as 0 in the quotient, $x^2$ is -1 in the quotient, then $x^4=1$, i.e., $x^2+\gen{x^2+1} = -1 + \gen{x^2+1}$.
	Then  e.g., $x^7-3x^5+x^3-x^2+2x-1+\gen{x^2+1} = -x-3x-x-(-1)+2x-1+\gen{x^2+1} = -3x+\gen{x^2+1}$.
	Thus, $\mb Z[x]/\gen{x^2+1} = \setm{a+bx+\gen{x^2+1}}{a,b\in\mb Z}$.
	$\setm{a+bi}{a,b\in\mb Z}$ such that $x^2=-1=i^2$.
	\say{$\mb Z[x]/\gen{x^2+1}\cong\mb Z[i]$}, \say{$\mb R[x]/\gen{x^2+1}\cong\mb C$.}
\end{example}

\begin{definition}
	Let $A$ be a $\os{\da A\neq R}{\text{proper}}$ ideal of commutative ring $R$. 
	Then $A$ is a \ul{prime ideal} if for every $a,b\in R$, if $ab\in A$ then $a\in A$ or $b\in A$.
\end{definition}
\begin{example}
	$\gen7$ is a prime ideal in $\mb Z$ since for $a,b\in\mb Z$ if $ab\in\gen7$ then $ab$ is a multiple of 7.
	So 7 is prime, so $7\mid a$ or $7\mid b$. Thus $a\in\gen7$ or $b\in\gen7$.
	More generally, if $p$ is prime, then $\gen p$ is a prime ideal in $\mb Z$.
\end{example}
\begin{example}
	[Non-example] $\gen6$ is not a prime ideal in $\mb Z$. $2,3\in\mb Z$, $2\cdot3=6\in\gen6$ but $2\notin\gen6$ and $3\notin\gen6$, so $\gen6$ is not prime.
\end{example}
\begin{question}
	is $\gen0$ prime in $\mb Z$? Yes, since $\gen0=\set0$. If $ab=0$ then $a=0$ or $b=0$ in $\mb Z$.
\end{question}
\begin{question}
	is $\gen0$ prime in $\mb Z_6$?
	$2,3\in\mb Z_6$, $2\cdot3=6$ in $\mb Z_6$ but $2\notin\gen0$ and $3\notin\gen0$.
\end{question}
\begin{definition}
	A proper ideal $A$ of a commutative ring $R$ is \ul{maximal} if for every ideal $B$ with $A\subseteq B\subseteq R$, we have $B=A$ or $B=R$.
\end{definition}
\begin{theorem}
	Let $R$ be a commutative ring with unity, and let $A$ be an ideal of $R$.
	\begin{itemize}
		\item[(i)] $R/A$ is an integral domain if and only if $A$ is prime.
		\item[(ii)] $R/A$ is a field if and only if $A$ is maximal.
	\end{itemize}
\end{theorem}
\begin{example}
	$R=\mb Z$, $A=5\mb Z$.

	$R=\mb Z[x]$, $A=\gen{x^2+1}$, $R/A\cong\mb Z[i]$,

	$R=\mb R[x]$, $A=\gen{x^2+1}$, $R/A\cong\mb C$.
\end{example}
\begin{proof}
	\begin{itemize}
		\item[(i)] exercise.
		\item[(ii)] Suppose $R/A$ is a field. Let $B$ be an ideal in $R$ such that $A\subseteq B\subseteq R$.
			\ul{Case 1}: $B=A$. done.
			\ul{Case 2}: $B\neq A$. Then there exists $b\in B$ such that $b\notin A$.
			Then $b+A\neq0+R$ i.e., $b+A$ is nonzero in $R/A$.
			Thus $b+A$ is a unit and so has an inverse, call it $c+A$.
			Then $(b+A)(c+A)=bc+A=1+A$. Then $bc-1\in A\subseteq B$, so $bc-1\in B$.
			Then $\us{\in B}{bc}-(\us{\in B}{bc-1})=1\in B$, $\therefore B=R \la$ exercise.

			Conversely, suppose $A$ is maximal. Let $b\in R,b\notin A$.
			Then $b+A\neq0+A$. We need to show $b+A$ is a unit.
			Let $B=\setm{rb+a}{a\in A,r\in R}$. Then
			\begin{itemize}
				\item[(i)] $B$ is an ideal of $R$ (exercise. use ideal test.)
				\item[(ii)] $B$ contains $A$ properly. (clear once you stare.)
			\end{itemize}
			Since $A$ is maximal, $B=R$. Then $1\in B$. There exists $r\in R$ and $a\in A$ such that $rb+a=1$.
			Then $(r+A)(b+A)=rb+A=1+A$, so $b+A$ is a unit. $\therefore R/A$ is a field.
	\end{itemize}
\end{proof}

\begin{question}
	Let $R$ be a commutative ring with unity. Let $A\subseteq R$ be an ideal.
	\begin{enumerate}
		\item If $A$ is maximal, is $A$ prime? (Yes. exercise)
		\item If $A$ is prime, is $A$ maximal? (No. produce counterexample)
	\end{enumerate}
	Counterexample for (2). Find ring $R$ and ideal $A$ such that $R/A$ is an integral domain but $R/A$ is not a field.
	$\mb Z$ is a ring that is an integral domain, but not a field.
	$A=\gen0=\set0$, $\mb Z/\gen0=\mb Z$.
	Counterexample: $R=\mb Z$, $A=\gen0$.
\end{question}

\section{Ring Homomorphisms}

\begin{definition}
	A \ul{(ring) homomorphism} is a map between rings $\varphi:R\ra S$, such that for every $a,b\in R$
	\begin{itemize}
		\item[(i)] $\varphi(a+b)=\varphi(a)+\varphi(b)$
		\item[(ii)] $\varphi(ab)=\varphi(a)\varphi(b)$
	\end{itemize}
	A bijective ring homomorphism is called a \ul{(ring) isomorphism}.
\end{definition}
\begin{example}
	$\varphi:\mb Z\ra\mb Z$, $\varphi(x)=x$
\end{example}
\begin{example}
	Fix positive integer $n$. $\varphi:\mb Z\ra\mb Z_n$, $\varphi(x)=x\mod n$, is a ring homomorphism.
\end{example}

I will grade test this weekend.

\ul{Last time}:
\begin{definition}
	A ring homomorphism s a map between rings $\varphi:R\ra S$ such that
	\begin{itemize}
		\item[(i)] $\varphi(a+b)=\varphi(a)+\varphi(b)$, for all $a,b\in R$
		\item[(ii)] $\varphi(ab)=\varphi(a)\varphi(b)$, for all $a,b\in R$.
	\end{itemize}
\end{definition}
For positive integer $n$, $\varphi:\mb Z\ra\mb Z_n$, $\varphi(x)=x\mod n$ \say{natural homomorphism $\mb Z$ to $\mb Z_n$}
\begin{xca}
	Confirm $\varphi$ is a homomorphism.
\end{xca}
\ul{Applications}: 1573435789\ul{6} even?
Decimal representation of $N\in\mb Z_{>0}$, $N=a_ka_{k-1}a_{k-2}\dots a_1a_0$ i.e., $N=\sum_{i=1}^k a_i 10^i$.
\begin{align*}
	\varphi_2(N) &= \varphi_2(\sum_{i=1}^k a_i10^i) \\
				 &= \sum_{i=1}^k \varphi_2(a_i)\varphi_2(10^i) \\
				 &= \sum_{i=1}^k \varphi_2(a_i)\varphi_2(10)^i = \varphi_2(a_0).
\end{align*}
\begin{note}
	$\varphi_2(10)=0$.
\end{note}
$\therefore \varphi_2(N)=\varphi_2(a_0$.
\begin{note}
	$N$ is divisible by 2 iff $\varphi_2(N)=\varphi_2(a_0)=0$.
\end{note}
\begin{xca}
	get divisibility rule for $n=3$. \ul{Note}: $\varphi_(10)=1$.
	$n=9$. \ul{Note}: $\varphi_9(1)=1$.
	$n=11$. \ul{Note}: $\varphi_{11}(10) = \varphi_{11}(-1)$.
	$n=4$. \ul{Note}: $\varphi_4(10) = 2$, $\varphi_4(10^2)=0$, $k\geq2$, $\varphi_4(10^k)=0$.
\end{xca}

\subsection{Properties of Ring homomorphisms}

Let $\varphi:R\ra S$ be a ring homomorphism.
\begin{enumerate}
	\item for every $r\in R$ and every $n\in\mb Z_{>0}$, $\varphi(n\cdot r)=n\cdot\varphi(r)$ and $\varphi(r^n)=\varphi(r)^n$.
	\item the image of a subring of $R$ is a subring of $S$ i.e., for every subring $A\subseteq R$, $\varphi(A)=\setm{\varphi(a)}{a\in A}$ is a subring of $S$.
		\begin{note}
			In particular, $im(\varphi)$ is a subring of $S$.
		\end{note}
	\item If $\varphi$ is surjective, the image of an ideal of $R$ is an ideal of $S$.
	\item The preimage of an ideal of $S$ is an ideal of $R$.
	\item if $R$ is commutative, then $im(\varphi)$ is commutative.
	\item if $R$ has unity, $S\neq\set0$, and $\varphi$ is surjective, then $\phi(1)=1$.
	\item $\varphi$ is injective if and only if $\ker\phi=\setm{r\in R}{\varphi(r)=0}$ is trivial i.e., $=\set0$.
	\item If $\varphi$ is an isomorphism, then $\varphi\inv:S\ra R$ is an isomorphism.
	\item $\ker\phi$ is an ideal of $R$.
\end{enumerate}
\begin{remark}
	if $\varphi:R\ra S$ is a ring homomorphism, then $\varphi:(R,+)\ra(S,+)$ is a group homomorphism.
\end{remark}
For (3), you can't remove the surjective hypothesis.
$R = \mb Z$, $S=\mb Z[x]$, $\varphi:\mb Z\ra\mb Z[x]$.
$\varphi(n)=n$. Consider the ideal $2\mb Z=\gen2$, then $\varphi(2\mb Z)=2\mb Z=\setm{2k}{k\in\mb Z}$.
$2\mb Z$ is \ul{not} an ideal in $\mb Z[x]$. $2\in2\mb Z$, $x\in\mb Z[x]$, but $2x\notin2\mb Z$.

\begin{theorem}
	Let $R$ be a ring with unity 1. Then $\varphi:\mb Z_{>0}\ra R$ with $n\mapsto n\cdot 1 = \ub{n\text{ times}}{1+1+\dots+1}$ can be extended in the obvious way to a map $\varphi\mb Z\ra R$ to be a ring homomorphism.
\end{theorem}
\begin{proof}
	[sketch] let $a,b\in\mb Z_{>0}$.
	\begin{align*}
		\varphi(a+b) &= (a+b)\cdot1 = \ub{a+b\text{times}}{1+1+\dots+1} \\
		\varphi(a)+\varphi(b) &= (\ub{a\text{ times}}{1+1+\dots+1}) + (\ub{b\text{ times}}{1+1+\dots+1})
	\end{align*}
	$\therefore \varphi(a+b)=\varphi(a)+\varphi(b)$ for $a,b\in\mb Z_{>0}$.
	\begin{align*}
		\varphi(ab) &= \ub{ab\text{ times}}{1+1+\dots+1}
					= \begin{rcases} \ob{a\text{ times}}{1+1+\dots+1} \\ + 1+1+\dots+1 \\ +\vdots \\ + 1+1+\dots+1 \end{rcases} b\text{ times}
	\end{align*}
\end{proof}
\begin{theorem}
	Let $R$ be a ring and let $A$ be an ideal of $R$. Then there exists a ring $S$ and a ring homomorphism $\varphi:R\ra S$ such that $\ker\phi=A$.
\end{theorem}
\begin{proof}
	[idea of proof] Let $S=R/S$. (Note: $S$ is a ring since $A$ is an ideal)
	Define $\varphi:R\ra R/A$ by $\varphi(r)=r+A$ for $r\in R$ is a ring homomorphism and \ul{$\ker\phi=A$} (\ul{exercise}: check!)
\end{proof}
\begin{enumerate}
	\item[$*$] How could we show $\ker\phi=A$?
		\begin{itemize}
			\item[(i)] Show $\ker\phi\subseteq A$
			\item[(ii)] Show $A\subseteq\ker\phi$
		\end{itemize}
\end{enumerate}

\begin{theorem}
	[First Isomorphism Theorem]
	Let $\varphi:R\ra S$ be a ring homomorphism.
	Then $\Phi:R/\ker\phi\ra im(\phi)$ given by $r+\ker\phi\mapsto\phi(r)$ is a ring isomorphism.
\end{theorem}
\begin{proof}
	First show $\Phi$ is well-defined. Suppose $r+\ker\phi=r'+\ker\phi$ for some $r,r'\in R$.
	Then $r'=r+t$ for some $t\in\ker\phi$.
	\begin{align*}
		\varphi(r') = \varphi(r+t) = \varphi(r)+\varphi(t) = \varphi(r)+0 = \varphi(r)
	\end{align*}
	$\therefore\Phi$ is well-defined. Next, show $\Phi$ is a ring homomorphism.
	Let $a+\ker\varphi$, $b+\ker\varphi\in R/\ker\phi$.
	\begin{align*}
		\Phi((a+\ker\varphi)+(b+\ker\varphi)) &= \Phi((a+b)+\ker\varphi) = \varphi(a+b) \\
		&= \varphi(a)+\varphi(b) = \Phi(a+\ker\varphi) + \Phi(b+\ker\varphi)
	\end{align*}
	multiplication (exercise). Finally, show $\Phi$ is a bijection.
	\ul{surjective}: let $b\in im(\varphi)$. There exists $r\in R$ such that $\varphi(r)=b$.
	Then $\Phi(r+\ker\varphi)=\varphi(r)=b$. $\therefore\Phi$ is surjective.
	\ul{injective}: option (1): let $a_1+\ker\varphi,a_2+\ker\varphi\in R/\ker\varphi$ with $\Phi(a_1+\ker\varphi)=\Phi(a_2+\ker\varphi)$.
	Then $\varphi(a_1)=\varphi(a_2)$. Then $\varphi(a_1)-\varphi(a_2)=0$ and $\varphi(a_1-a_2)=0$. $\therefore a_1-a_2\in\ker\varphi$, $\therefore a_1+\ker\varphi=a_2+\ker\varphi$. $\therefore \Phi$ is injective.
	\ul{option 2}: Show $\ker\Phi=\set{\ker\varphi}$ (exercise).
	Thus $\Phi$ is a ring isomorphism.
\end{proof}
\begin{example}
	Fix $n>0$ integer. $\varphi_n:\mb Z\ra\mb Z_n$, then $\ker(\varphi_n)=n\mb Z=\gen n$ and $im(\varphi_n)=\mb Z_n$ since $\varphi_n$ is surjective.
	$\therefore$ by first isomorphism theorem, $\mb Z/n\mb Z\cong\mb Z_n$.
\end{example}

\subsection{Field of Quotients (Fraction field)}
Let $R$ be an integral domain. Then there exists a field $F$ that contains a subring isomorphic to $R$.
\begin{example}
	[$\mb Z= R$, $\mb Q = F$]
	$\mb Q=\setm{\frac ab}{a,b\in\mb Z,b\neq0,\dots}$, $\mb Z\not\subset\mb Q$, $\frac ab+\frac cd=\frac{ad+bc}{bd}$, $\mb Z\us{inj}\lra\mb Q$, $n\mapsto\frac n1$.

	Let $S=\setm{(a,b)}{a,b\in R,b\neq0}$, impose equivalence relation on $S$ such that $(a,b)\sim(c,d)$ if and only if $ad-bc=0$ or equivalently iff $ad=bc$.
	Denote the equivalence class of $(a,b)$ by $\frac ab$.
	\begin{note}
		$\frac ab=\frac cd$ if and only if $ad=bc$.
	\end{note}
	Define addition and multiplication on the equivalence classes: $\frac ab+\frac cd=\frac{ad+bc}{bd}$ and $\plr{\frac ab}\plr{\frac cd}=\frac{ac}{bd}$.
	We need to
	\begin{enumerate}
		\item show operations are well-defined
		\item Show $S/\sim$ with addition and multiplication as above is a field.
		\item let $\varphi:R\ra S/sim$ be defined y $\varphi(r)=\frac r1$. Show $\varphi$ is an injective homomorphism.
			\begin{note}
				Then $im(\varphi)\cong R/\ker\varphi\cong R$, 1st isomorphism theorem.
			\end{note}
	\end{enumerate}
	Show operations are well-defined.
	Suppose $\frac ab=\frac{a'}{b'}$ and $\frac cd=\frac{c'}{d'}$.
	Then $ab'=a'b$ and $cd'=c'd$.
	So $\plr{\frac{a'}{b'}}\plr{\frac{c'}{d'}}=\frac{a'c'}{b'd'}$ and $\plr{\frac ab}\plr{\frac cd}=\frac{ac}{bd}$.
	Therefore $a'c'bd=a'bc'd=ab'c'd=ab'cd'=acb'd'$.
	Thus $\plr{\frac{a'}{b'}}\plr{\frac{c'}{d'}}=\plr{\frac ab}{\plr{\frac cd}}$.

	\ul{addition}: Check at home that $\frac{a'}{b'}+\frac{c'}{d'}=\frac ab+\frac cd$.
	$S/\sim$ with addition and multiplication is a field $F$ (exercise).
\end{example}

\begin{theorem}
	[Steinitz 1910] If $F$ is a field of characteristic $p$ (prime), then $F$ contains a subfield isomorphic to $\mb F_p$. If $char(F)=0$, $F$ contains a subfield isomorphic to $\mb Q$.
\end{theorem}
\begin{proof}
	[Sketch] $\varphi:\mb Z\ra F$, $n\mapsto n\cdot 1$ is a ring homomorphism.
	By 1st isomorphism theorem $im(\varphi)\cong\mb Q/\ker\varphi$.
	If $char(F)$ is $p$ prime, then $\ker\varphi=\gen p=p\mb Z$.
	Then $\mb Z/p\mb Z\cong\mb F_p$, so $S=im(\varphi)\subseteq F$ is the subfield isomorphic to $\mb F_p$.
	If $char(F)=0$, then $\ker\varphi=\set0$ so $F$ contains a subring isomorphic to $\mb Z$.
	Let $T=\setm{ab\inv}{a\in S,b\in S,b\neq0}$
	\begin{xca}
		Show $T$ is a field.
	\end{xca}
\end{proof}
\begin{remark}
	$\mb F_p$ is the finite field with $p$ elements $\mb Z_p=\set{0,1,2,\dots,p-1}$ with arithmetric $\mod p$.
\end{remark}

\section{Polynomial Rings}
Let $R$ be a commutative ring. The \ul{ring of polynomials over $R$ in indeterminate $x$}, denoted $R[x]$, is the set
$R[x] = \setm{a_nx^n+a_{n-1}x^{n-1}+\dots+a_1x+a_0}{a_i\in R,n\in\mb Z_{\geq0}}$.
Two polynomials $a_nx^n+\dots+a_1x+a_0$ and $b_mx^m+\dots+b_1x+b_0$ are equal if $a_i=b_i$ for all $i\in\mb Z_{\geq0}$, where
\begin{itemize}
	\item $a_i$ is defined to be 0 for $i>n$,
	\item $b_i$ is defined to be 0 for $i>m$.
\end{itemize}
endow $R[x]$ with addition and multiplication:
\begin{itemize}
	\item $\sum_{i=0}^n a_ix^i+\sum_{j=0}^m b_jx^j = \sum_{k=0}^{\max(n,m)} (a_k+b_k)x^k$ (using same convention ofr $a_i,b_i$, $i,j$ large)
	\item $\plr{\sum_{i=0}^na-ix^i}\plr{\sum_{j=0}^mb_jx^j} = \sum_{k=0}^{n+m}c_kx^k$, where $c_k = a_0b_k+a_1b_{k-1}+\dots+a_kb_0$ (FOIL)
\end{itemize}
let $f(x)=a_nx^n+\dots+a_1x+a_0$. if $a_n\neq0$, we say the degree of $f(x)$ is $n$, and we say $a_n$ is the \ul{leading coefficient}.
\begin{note}
	The zero polynomial has no degree. if leading coefficient of $f(x)$ is 1, we say $f(x)$ is monic.
\end{note}
\begin{example}
	Polynomials are different from polynomial functions.
	Consider for $\mb Z_3[x]$ the polynomial $g(x)=x(x+1)(x+2)=x(x^2+0x+2)=x^3+2x$ and the function $f:\mb Z_3\ra\mb Z_3$ with $f(x)=0$.
	Interpret $g(x)$ as a polynomial function $\mb Z_3$ to $\mb Z_3$, $g(0)=0^3+2(0)=0$, $g(1)=1^3+2(1)=0$, $g(2)=2^3+2(2)=0$.
\end{example}
\begin{theorem}
	If $R$ is an integral domain then $R[x]$ is an integral domain.
\end{theorem}
\begin{proof}
	[sketch] We already have $R[x]$ is a ring. Since $R$ is commutative, $R[x]$ is commutative (exercise).
	Let $1\in R$ be unity. Then constant polynomial $f(x)=1$ is unity in $R[x]$.
	Let $f(x)=a_nx^n+\dots+a_1x+a_0$ and $g(x)=b_mx^m+\dots+b_1x+b_0$ be elements of $R[x]$, with $a_n\neq0$ and $b_m\neq0$.
	The leading coefficient of $f(x)g(x)$ is $a_mb_m\neq0$ since $a_n\neq0,b_m\neq0$ and $R$ is an integral domain.
	Therefore $f(x)g(x)$ is not zero polynomial. Thus $R[x]$ is an integral domain.
\end{proof}
\begin{theorem}
	[Division Alrogithm for Polynomial Field]
	Let $F$ be a field, let $f(x),g(x)\in F[x]$, with $g(x)\neq0$.
	Then there \ul{\ul{exists}} \ul{unique} polynomials $q(x)$ and $r(x)$ such that $f(x)=g(x)q(x)+r(x)$, where $r(x)=0$ or $\deg(r(x))<\deg(g(x))$. ($q(x)$ is the quotient and $r(x)$ is the remainder)
\end{theorem}
\begin{example}
	$f(x)=3x^3-wx^2+x+1$, $g(x)=x^2+3x-1$
	\begin{align*}
		x^2 + 3x - 1 &\mid 3x^3-2x^2+x+1 \\
					 &- (3x^3+9x^2-3x) \\
					 &= -11x^2+4x+1 \\
					 &- (-11x^2-33x+11) \\
					 &= 37x-10
	\end{align*}
	Remainder $r(x)=37x-10$ and quotient $q(x)=3x-11$.
	\ul{check}: $f(x)=g(x)q(x)+r(x)$.
\end{example}
\begin{corollary}
	[Remainder Theorem] Let $F$ be a field. Let $f(x)\in F[x]$. Let $a\in F$.
	Then $f(a)$ is equal to the remainder of $f(x)$ divided by $(x-a)$.
\end{corollary}
\begin{proof}
	By Division algorithm, there exists unique $q(x),r(x)$ such that $f(x)=(x-a)q(x)+r(x)$, where $r(x)=0$ or $\deg(r(x))<\deg(x-a)=1$ i.e., $r(x)=0$ or $r(x)$ is constant nonzero.
	Plug in $x=a$ to $f$, so $f(a)=(a-a)q(a)+r(a)=0\cdot q(a)+r(a)=r(a)$.
	Then $r(a)=f(a)$ and $r(x)=f(a)$, since $r(x)$ is constant.
\end{proof}
\begin{corollary}
	[Factor Theorem] $x-a$ is a factor of $f(x)$ if and only if $a$ is a zero of $f(x)$ i.e., $f(a)=0$.
\end{corollary}
\begin{proof}
	exercise.
\end{proof}

Test 2: Tues, covers chapters 14,15
\begin{itemize}
	\item Ideals and Factor Rings
	\item Ring Homomorphisms
\end{itemize}
No Office hours Tues (email or post).
\begin{theorem}
	A polynomial of degree $n$ over a field $f$ has at most $n$ zeros, counting multiplicity.
\end{theorem}
\begin{proof}
	By induction on $n$.
	\ul{Base case} (n=0) These polynomials are the nonzero constants, hence they have no zeros.
	\ul{Inductive step}: Fix $n\geq1$ and assume result for polynomials of degree less than $n$.
	Let $f(x)\in F[x]$ be a polynomial of degree $n$.
	If $f(x)$ has no zeros, then done. $(0<n)$.
	Otherwise, $f(x)$ has a zero. Let $a\in F$ be a zero.
	Then $x-a$ is a factor of $f(x)$. If $a$ has multiplicity $k$ as a root, then $f(x)=(x-a)^kq(x)$ for some polynomial $q(x)\in F[x]$.
	\begin{note}
		$n=\deg(f(x))=\deg((x-a)^kq(x))=\deg((x-a)^k)+\deg(q(x)) = k+\deg(q(x))$
	\end{note}
	Since $k\geq1$, $\deg(q(x))<n$. Additionally, $k\leq n$.
	If $f(x)$ has no other zeros, then done. Otherwise let $b\neq a$ be a zero of $f(x)$.
	Then $0=f(b)=(b-a)^kq(b)$. Since $F$ is a field, this implies $q(b)=0$.
	In particular, $b$ is a zero of $q(x)$ \ul{and} the multiplicity of $b$ is a root of $q(x)$ is the same as the multiplicity of $b$ as a root of $f(x)$.
	Since $\det(q(x))<n$, by the inductive hypothesis $q(x)$ has at most $n-k$ roots, counting multiplicity.
	Thus $f(x)$ has at most $k+(n-k)=n$ roots, counting multiplicity.
	By induction, done.
\end{proof}

We cannot relax $F$ field hypothesis
\begin{example}
	for $x^2+x\in\mb Z_6[x]$, $f(0)=0^2+0=0$, $f(2)=2^2+2=0$, $f(3)=3^2+3=0$
\end{example}
\begin{question}
	What about polynomials over an integral domain?
\end{question}
\begin{recall}
	If $R$ is a commutative ring and $a\in R$, then $\gen a=aR=\setm{ra}{r\in R}$ is \ul{principal ideal} generated by $a$.
\end{recall}
\begin{definition}
	A \ul{principal ideal domain} (PID) is an integral domain $R$ such that every ideal has the form $\gen a=\setm{ra}{r\in R}$ for some $a\in R$.
\end{definition}
\begin{example}
	$\mb Z$ is PID, for $n\in\mb Z$, $\gen{n}=n\mb Z=\setm{km}{k\in\mb Z}$.
	Take $\gen{3,5}\subseteq\mb Z$, which is $\setm{3s+5t}{s,t\in\mb Z}=\gen1=\mb Z$ in general for $a,b\in\mb Z$, $\gen{a,b}=\gen{\gcd(a,b)}$.
\end{example}

\begin{theorem}
	Let $F$ be a field. Then $F[x]$ is a PID.
\end{theorem}
\begin{proof}
	Since $F$ is a field, $F$ is an integral domain. Thus $F[x]$ is an integral domain.
	Thus $F[x]$ is an integral domain. Let $I\subseteq F[x]$ be an ideal.
	If $I=\set0$, then $I=\gen0$, so $I$ is principal.
	Otherwise, let $g(x)\in I$ be a polynomial of minimal degree.
	We want to show $\gen{g(x)}=I$. It's clear that $\gen{g(x)\subseteq I}$.
	Let's show $I\subseteq\gen{g(x)}$. Let $f(y)\in I$.
	By the division algorithm, there exist unique polynomials $q(x)$ and $r(x)\in F[x]$ such that $f(x)=g(x)q(x)+r(x)$, and $r(x)=0$ or $\deg(r(x))<\deg(g(x))$.
	\begin{note}
		$r(x)=\ub{\in I}{\us{\in I}{f(x)}-\ub{\in I}{\us{\in I}{g(x)}q(x)}}\in I$.
	\end{note}
	Since $g(x)$ is chosen to have minimal degree, we must have $r(x)=0$.
	Then $f(x)=g(x)q(x)\in\gen{g(x)}$. Thus $I\subseteq\gen{g(x)}$.
\end{proof}

\newpage

\begin{theorem}
	Let $F$ be a field. Let $I\subseteq F[x]$ be a nonzero ideal. Let $g(x)\in F[x]$.
	Then $I=\gen{g(x)}$ if and only if $g(x)$ is a nonzero polynomial of minimal degree in $I$.
\end{theorem}
\begin{example}
	Consider the ring homomorphism $\varphi:\mb R[x]\ra\mb C$ with $f(x)\mapsto f(i)$.
	What is $\ker\varphi$? Note $\ker\varphi$ is an ideal in $\mb R[x]$.
	It's clear that $x^2+1\in\ker\varphi$. It's clear that no nonzero constant function ($\deg 0$), no linear polynomial ($\deg1$ are in $\ker\varphi$.
	Thus $x^2+1$ is a polynomial of minimal degree in $\ker\varphi=\gen{x^2+1}$.
	Claim: $\varphi$ is surjective. Let $a+bi\in\mb C$. Then $\varphi(a+bx)=a+bi$.
	Thus $\varphi$ is surjective.
	Then by the first isomorphism theorem, $\mb C\cong\mb R[x]/\gen{x^2+1}$.
	\begin{note}
		$\gen{x^2+1}$ is a prime ideal since $\mb C$ is an integral domain.
		$\gen{x^2+1}$ is a maximal ideal since $\mb C$ is a field.
	\end{note}
\end{example}
\begin{example}
	[$R=\mb Z$] $\gen4\cap\gen{20}=\gen g$ for some $g\in\mb Z$, and $g=20$.
	Take $\gen4\cap\gen{10}=\gen{20}$.
\end{example}
In general, for $a,b\in\mb Z$,
\begin{align*}
	\gen a\cap\gen b &=\gen{\text{lcm}(a,b)} \\
	\gen{a,b}=\gen{\gcd(a,b)} \\
	\gen a+\gen b = \, ? \\
	\gen a\gen b = \, ?
\end{align*}
\ul{general}
if $A,B$ are ideals in a common ring $R$,
\begin{align*}
	A+B &= \setm{a+b}{a\in A,b\in B} \\
	AB &= \setm{\sum a_ib_i}{a_i\in A,b_i\in B}
\end{align*}
\begin{example}
	[$R=\mb Z$] $\gen4+\gen{10} = \setm{a+b}{a\in\gen4,b\in\gen{10}} = \setm{4s+10t}{s,t\in\mb Z} = \gen{\gcd(4,10)} = \gen2$
\end{example}
for $a,b\in\mb Z$, $\gen a+\gen b=\gen{\gcd(a,b)}$.
\begin{align*}
	\gen4\gen{10} &= \setm{\sum a_ib_i}{a_i\in\gen4,b_i\in\gen{10}} = \setm{\sum4s_i10t_i}{s_i\in\mb Z,t_i\in\mb Z} = \setm{\sum40r_i}{r_i\in\mb Z}=\gen{40}.
\end{align*}
In general, $\gen a\gen b=\gen{ab}$. exercise.

\begin{definition}
	Let $R$ be integral domain. A nonzero, nonunit polynomial $f(x)\in R[x]$ is \ul{irreducible over $R$} if whenever $f(x)=g(x)h(x)$ with $g(x),h(x)\in R[x]$, then $g(x)$ or $h(x)\in R[x]^\times$.
\end{definition}
\begin{example}
	$f(x)=2x^2+4$
	\begin{itemize}
		\item factors to $f(x)=2(x^2+2)$, so $f(x)$ is reducible over $\mb Z$.
			(Note: $2,(x^2+2)\notin\mb 1[x]^\times$)
		\item $f(x)=2x^2+4$ is irreducible over $\mb Q$ (2 is a unit in $\mb Q$)
			\begin{itemize}
				\item irreducible over $\mb R$
				\item reducible over $\mb C$
			\end{itemize}
	\end{itemize}
\end{example}
\ul{intuition}: think about polynomials of degree 2,3, if there are no roots then it is irreducible.

\subsection*{Reducibility Test for degree 2 and 3}
Let $F$ be a \ul{field}. Let $f(x)\in F[x]$ be a polynomial of degree 2 or 3.
Then $f(x)$ is reducible over $F$ if and only if $f(x)$ has a zero in $F$.
\begin{proof}
	(later).
\end{proof}
\begin{example}
	$f(x)=x^3+x+1\in\mb Z_2[x]$
	\begin{note}
		$\mb Z_2$ is a field.
	\end{note}
	$f(0)=0^3+0+1=1\neq0$, $f(1)=1^3+1+1=1\neq0$, $\therefore f(x)$ has no zeros so by the reducibility test, $f(x)$ is irreducible over $\mb Z_2$.
\end{example}

\end{document}
