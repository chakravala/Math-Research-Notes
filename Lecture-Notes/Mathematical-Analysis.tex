\documentclass[]{article}
\usepackage[latin1]{inputenc}
\usepackage{graphicx}
\usepackage[left=1.00in, right=1.00in, top=1.10in, bottom=1.00in]{geometry}

\usepackage{dirtytalk}
\usepackage[normalem]{ulem}
\usepackage{tikz-cd}
\usepackage{units}
\usepackage{algorithm}
\usepackage{algpseudocode}
\usepackage{alltt}
\usepackage{mathrsfs}
\usepackage{amssymb}
\usepackage{amsmath}
\DeclareMathOperator\cis{cis}

% (font shortcuts)
\usepackage{amsfonts}
\newcommand{\mb}[1]{\mathbb{#1}}
\newcommand{\mc}[1]{\mathcal{#1}}
\newcommand{\ms}[1]{\mathscr{#1}}
\newcommand{\mf}[1]{\frak{#1}}

% (arrow shortcuts)
\newcommand{\ra}{\rightarrow}
\newcommand{\lra}{\longrightarrow}
\newcommand{\la}{\leftarrow}
\newcommand{\lla}{\longleftarrow}
\newcommand{\Ra}{\Rightarrow}
\newcommand{\Lra}{\Longrightarrow}
\newcommand{\La}{\Leftarrow}
\newcommand{\Lla}{\Longleftarrow}
\newcommand{\lr}{\leftrightarrow}
\newcommand{\llr}{\longleftrightarrow}
\newcommand{\Lr}{\Leftrightarrow}
\newcommand{\Llr}{\Longleftrightarrow}

% (match parenthesis)
\newcommand{\mlr}[1]{\left|#1\right|}
\newcommand{\plr}[1]{\left(#1\right)}
\newcommand{\blr}[1]{\left[#1\right]}

% (exponent shortcuts)
\newcommand{\inv}{^{-1}}
\newcommand{\nrt}[2]{\sqrt[\leftroot{-2}\uproot{2}#1]{#2}}

% (annotation shortcuts)
\newcommand{\conj}[1]{\overline{#1}}
\newcommand{\ol}[1]{\overline{#1}}
\newcommand{\ul}[1]{\underline{#1}}
\newcommand{\os}[2]{\overset{#1}{#2}}
\newcommand{\us}[2]{\underset{#1}{#2}}
\newcommand{\ob}[2]{\overbrace{#2}^{#1}}
\newcommand{\ub}[2]{\underbrace{#2}_{#1}}
\newcommand{\bs}{\backslash}
\newcommand{\ds}{\displaystyle}

% (set builder)
\newcommand{\set}[1]{\left\{ #1 \right\}}
\newcommand{\setc}[2]{\left\{ #1 : #2 \right\}}
\newcommand{\setm}[2]{\left\{ #1 \, \middle| \, #2 \right\}}

% (group generator)
\newcommand{\gen}[1]{\langle #1 \rangle}

% (functions)
\newcommand{\im}[1]{\text{im}(#1)}
\newcommand{\range}[1]{\text{range}(#1)}
\newcommand{\domain}[1]{\text{domain}(#1)}
\newcommand{\dist}[1]{(#1)}
\newcommand{\sgn}{\text{sgn}}

% (Linear Algebra)
\newcommand{\mat}[1]{\begin{bmatrix}#1\end{bmatrix}}
\newcommand{\pmat}[1]{\begin{pmatrix}#1\end{pmatrix}}
%\newcommand{\dim}[1]{\text{dim}(#1)}
\newcommand{\rnk}[1]{\text{rank}(#1)}
\newcommand{\nul}[1]{\text{nul}(#1)}
\newcommand{\spn}[1]{\text{span}\,#1}
\newcommand{\col}[1]{\text{col}(#1)}
%\newcommand{\ker}[1]{\text{ker}(#1)}
\newcommand{\row}[1]{\text{row}(#1)}
\newcommand{\area}[1]{\text{area}(#1)}
\newcommand{\nullity}[1]{\text{nullity}(#1)}
\newcommand{\proj}[2]{\text{proj}_{#1}\left(#2\right)}
\newcommand{\diam}[1]{\text{diam}\,#1}

% (Vectors common)
\newcommand{\myvec}[1]{\vec{#1}}
\newcommand{\va}{\myvec{a}}
\newcommand{\vb}{\myvec{b}}
\newcommand{\vc}{\myvec{c}}
\newcommand{\vd}{\myvec{d}}
\newcommand{\ve}{\myvec{e}}
\newcommand{\vf}{\myvec{f}}
\newcommand{\vg}{\myvec{g}}
\newcommand{\vh}{\myvec{h}}
\newcommand{\vi}{\myvec{i}}
\newcommand{\vj}{\myvec{j}}
\newcommand{\vk}{\myvec{k}}
\newcommand{\vl}{\myvec{l}}
\newcommand{\vm}{\myvec{m}}
\newcommand{\vn}{\myvec{n}}
\newcommand{\vo}{\myvec{o}}
\newcommand{\vp}{\myvec{p}}
\newcommand{\vq}{\myvec{q}}
\newcommand{\vr}{\myvec{r}}
\newcommand{\vs}{\myvec{s}}
\newcommand{\vt}{\myvec{t}}
\newcommand{\vu}{\myvec{u}}
\newcommand{\vv}{\myvec{v}}
\newcommand{\vw}{\myvec{w}}
\newcommand{\vx}{\myvec{x}}
\newcommand{\vy}{\myvec{y}}
\newcommand{\vz}{\myvec{z}}
\newcommand{\vzero}{\myvec{0}}

%\usepackage[active,tightpage]{preview}
\setlength\PreviewBorder{7.77pt}
\usepackage{varwidth}
\AtBeginDocument{\begin{preview}\begin{varwidth}{\linewidth}}
\AtEndDocument{\end{varwidth}\end{preview}}


\author{Book: Walter Rudin 3rd, Presenter: Maya Chhetri, Notes by Michael Reed}
\title{Mathematical Analysis}
%date{}

\begin{document}
\maketitle

%\begin{abstract}
%\end{abstract}

%\ul{Chapter 3 continued}: 

\begin{recall}
	[Cauchy product] $\sum_{n=0}^\infty a_n$, $\sum_{n=0}^\infty b_n$, then Cauchy product series is:
	$\sum_{n=0}^\infty c_n$, 
	$$c_n = \sum_{k=0}^n a_kb_{n-k} = a_0b_n + a_1b_{n-1} + \dots + a_nb_0.$$
\end{recall}

\begin{theorem}
	\label{thm-3-50}
	Suppose
	\begin{enumerate}
		\item[(a)] $\sum_{n=0}^\infty a_n$ converges absolutely;
		\item[(b)] $\sum_{n=0}^\infty a_n = A$
		\item[(c)] $\sum_{n=0}^\infty b_n = B$
		\item[(d)] $c_n: = \sum_{k=0}^n a_kb_{n-k}$.
	\end{enumerate}
	Then $\sum_{n=0}^\infty c_n$ converges to $AB$.
\end{theorem}
\begin{proof}
	Define sequence of partial sums $A_n:= \sum_{k=0}^n a_k$, $B_n:=\sum_{k=0}^nb_k$, $C_n:=\sum_{k=0}^n c_k$.
	By assumption: $A_n\lra A$ and $B_n\lra B$.
	\ul{Want to show}: $C_n\lra AB$.
	Then 
	\begin{align*}
		C_n&:=\sum_{k=0}^nc_k = c_0+c_1+c_2+\dots+c_n \\
		   &= a_0b_0+(a_0b_1+a_1b_0) + (a_0b_2+a_1b_1+a_2b_0) + \dots + (a_0b_n + a_1b_{n-1}+\dots+a_nb_0) \\
		   &= a_0(\ob{B_n}{b_0 + b_1+\dots+b_n}) + a_1(\ob{B_{n-1}}{b_0+\dots+b_{n-1}}) + a_2(\ob{B_{n-2}}{b_0+\dots+b_{n-2}}) + \dots + a_n\ob{B_0}{b_0} \\
		   &= a_0B_n + a_1b_{n-1} + a_2B_{n-2} + \dots + a_nB_0 
		   = a_0(\beta_n+B) + a_1(\beta_{n-1}+B) + \dots + a_n(\beta_0 + B) \\
		   &= (\ub{A_n}{a_0+a_1+\dots+a_n})B + a_0\beta_n + a_1\beta_{n-1} + \dots + a_n\beta_0 \lra AB
	\end{align*}
	if $a_0\beta_n+a_1\beta_{n-1}+\dots+a_n\beta_0\ra 0$ as $n\ra\infty$, where $\beta_n = B_n-B$ and $B_n=\beta_n+B$.
	
	Let $\gamma_n:= \alpha_0\beta_n + \alpha_1\beta_{n-1} + \dots + a_n\beta_0$.
	\ul{Fix $\epsilon>0$}. \ul{\ul{NTS}}: $\exists N\in\mb N$ such that $|\gamma_n|<\epsilon$ for all $n\geq N$.
	Since $\sum a_n$ converges absolutely, $\alpha:= \sum_{n=0}^\infty |a_n|\in\mb R$.
	Then $|\gamma_n|=|a_0\beta_n + a_1\beta_{n-1} + \dots + a_n\beta_0|$.
	Since $\beta_n\lra 0$ as $n\ra\infty$, $\exists N_1\in\mb N$ such that $|\beta_n|<\epsilon$ for all $n\geq N_1$.
	For $n\geq N_1$,
	\begin{align*}
		|\gamma_n|&=|a_0\beta_n+\dots+a_{n-N_1-1}\beta_{N_1+1} + a_{n-N_1}\beta_{N_1} + \dots + a_n\beta_0| \\
				  &\os{\Delta}{\leq} |a_0\beta_n + \dots + a_{n-N_1-1}\beta_{N_1+1}| + |a_{n-N_1}\beta_{N_1} + \dots + a_n\beta_0| \\
				  &\os{\Delta}{\leq} |a_0|\os{<\epsilon}{|\beta_n|}+|a_1|\os{<\epsilon}{|\beta_{n-1}|}+\dots + |a_{n-N_1-1}|\os{<\epsilon}{|\beta_{N_1+1}|} + |a_{n-N_1}\beta_{N_1} + \dots + a_n\beta_0| \\
				  &< \epsilon \ub{<\alpha}{\sum_{k=0}^{n-N_1-1} |a_k|} + |a_{n-N_1}\beta_{N_1} + \dots + a_n\beta_0| 
				  < \epsilon\alpha + |a_{n-N_1}\beta_{N_1} + \dots + a_n\beta_0|
	\end{align*}
	Keep $N_1$ fixed.
	Letting $n\ra\infty$ and noting that $a_i\ra0$ as $i\ra\infty$, we get $\lim_{n\ra\infty} |\gamma_n|\leq\epsilon\alpha$.
	Since $\alpha>0$ is fixed and $\epsilon>0$ is arbitrary, $\lim_{n\ra\infty} |\gamma_n| = 0$.
\end{proof}

\begin{theorem}
	\label{thm-3-51}
	If $\sum a_n = A$, $\sum b_n = B$, $\sum c_n=C$, where $c_n=\sum_{k=0}^n a_kb_{n-k}$.
	Then $C=AB$.
\end{theorem}
\begin{proof}
	-- In Chapter 8.
\end{proof}

\subsection*{Rearrangement of an infinite series}

$\sum a_n$ is $\sum a_n'$, where $a_n'=a$ where $k_n$ is a bijection from $\mb N$ to $\mb N$.
\begin{example}
	$a_1\ a_2\ a_3\ a_4\ a_5\ a_6\ a_7\ \cdots$, 
	
	$a_1\ a_3\ a_2\ a_5\ a_7\ a_4\ \cdots$ (rearrangement),
	
	$a_1\ a_3\ a_5\ a_7\ \cdots\ a_{2n+1}\ \cdots\ a_2\ a_4\ a_6\ a_8\ \cdots$ (not a rearrangement).
\end{example}
\begin{example}
	Alternating harmonic series: $\sum_{n=1}^\infty (-1)^{n-1}\frac1n = 1-\frac12+\frac13-\frac14+\frac15-\dots = \ln 2$.
	
	Rearrange: 2 odd terms followed by even term $1+\frac13-\frac12+\frac15+\frac17-\frac14+\frac19+\frac1{11}-\frac16+\dots$.
	
	\ul{\ul{Show}}: Sum of rearranged series is $\frac32\ln2$.
	$$\ln 2 = 1-\frac12+\frac13-\frac14+\frac15-\frac16+\frac17-\frac18+\dots.$$
	Multiply by $\frac12$:
	$$\frac{\ln2}{2} = 0+\frac12+0-\frac14+0+\frac16+0+\dots.$$
	Add the two above:
	$$ \frac32\ln2 = 1+0+\frac13-\frac12 + \frac15 + \frac17-\frac14+\dots,$$
	so rearranged series sum is $\frac32\ln 2$.
\end{example}

\begin{theorem}
	Suppose $\sum a_n$ converge absolutely.
	Then \ul{every} rearrangement $\sum a_n'$ of $\sum a_n$ converges, and they all converge to the same sum.
\end{theorem}
\begin{proof}
	Let $\sum a_n'$ be a rearrangement of $\sum a_n$.
	Let $s_n$ and $s_n'$ be sequences of partial sums of $\sum a_n$ and $\sum a_n'$ respectively.
	Also $s_n\lra s$ (say). \ul{\ul{NTS}}: $s_n'\ra s$.
	Let $\epsilon>0$ be fixed.
	\ul{NTS}: $\exists N\in\mb N$ such that $|s_n'-s|<\epsilon$ for all $n\geq N$.
	Now, 
	\begin{align*}
		|s_n'-s|&=|s_n'-s_n+s_n-s| \\
				&\leq |s_n'-s_n|+|s_n-s|
	\end{align*}
	Since $\sum a_n$ converges absolutely, $\exists N_2\in\mb N$ such that $m\geq n>N_2$
	$$\sum_{i=n}^m |a_i| < \frac\epsilon2.$$
	Choose $p\in\mb N$ such that $\set{1,2,\dots,N_2}\subset\set{k_1,\dots,k_p}$.
	Then for $n>p$
	\begin{align*}
		|s_n'-s_n| &= |(a_1'+a_2'+\dots+a_n')-(a_1+a_2+\dots+a_n)| &\qquad \text{-- \ul{cancel terms}} \\
				  &\leq \sum_{i=n}^{m} |a_i| < \frac\epsilon2
	\end{align*}
\end{proof}

\section*{Continuity}

\ul{From Calculus}: let $f:\mb R\ra\mb R$ such that $f$ may not be  defined at $p$, then $$\lim_{x\ra p} f(x) = q \iff \forall\epsilon>0,\exists \delta=\delta(\epsilon,p)>0: (0<|x-p|<\delta \implies |f(x)-q|<\epsilon).$$
$f$ is continuous at $p\in\mb R$ if $\lim_{x\ra p} f(x) = f(p)$ if and only if $\forall\epsilon>0,\exists\delta=\delta(\epsilon,p)>0$ such that $\forall x\in\mb R$ with $|x-p|<\delta$ implies $|f(x)-f(p)|<\epsilon$.
\ul{In metric spaces}: Let $f:E\subset(X,d_X)\ra(Y,d_Y)$.
\begin{definition}
	[Limit] Let $p\in E'$. We say $\lim_{x\ra p} f(x) = q$ if and only if $\forall \epsilon>0,\exists\delta=\delta(\epsilon,p)>0$ such that $\forall x\in E$ with $0<d_X(x,p)<\delta$ implies $d_Y(f(x),q) < \epsilon$.
	So $$X\supset E\supset B_\delta(p) \os{f}{\lra} f(B_\delta(p))\subset Y.$$
\end{definition}

Alternative definition of limit.

\begin{theorem}
	Let $f:E\subset(X,d_X)\ra(Y,d_Y)$ and $p\in E'$.
	Then $\lim_{x\ra p} f(x)=q \iff \lim_{n\ra\infty} f(x_n) = q$ for every sequence $\set{x_n}\subset E$ with $x_n\neq p$ and $x_n\ra p$.
\end{theorem}
\begin{proof}
	\say{$\implies$} Suppose $\lim_{x\ra p} f(x) = q$. Let $\set{x_n}\subset E$ such that $x_n\neq p$ and $x_n\ra p$.
	\ul{\ul{NTS}}: $\lim_{n\ra\infty} f(x_n) = p$.
	Let $\epsilon>0$ be fixed. 
	(NTF: $N\in\mb N$ such that $d_Y(f(x_n),q)<\epsilon$ for all $n>N$).
	Since $x_n\ra p \implies \exists N\in\mb N$ such that $d_X(x_n,p)<\delta$ for $n>N$.
	Since $\lim_{x\ra p} f(x) = q$, $\exists\delta=\delta(\epsilon,p)>0$ such that $\forall x\in E,x\neq p$ then $0<d_X(x,p)<\delta\implies d_Y(f(x),q)<\epsilon$.
	Therefore, for $n>N$, $d_x(x_n,p)<\delta$ which implies $d_Y(f(x_n),q)<\epsilon \implies \lim_{n\ra\infty}f(x_n)=q$.

	\say{$\Lla$} Let $\lim_{n\ra\infty} f(x_n) = q$ for \ul{every} sequence $\set{x_n}\subset E$, $x_n\neq p$ and $x_n\ra p$.
	Suppose by contradiction that: $\lim_{x\ra p} f(x)\neq q$.
	This means, $\exists\epsilon>0$ such that \ul{$\forall\delta>0$}, $\exists x_\delta\in E$ such that $0<d_X(x_\delta,p)<\delta$ but $d_Y(f(x),q)\geq \epsilon$.
	Taking $\delta_n = \frac1n$ and $x_n = x_{\delta_n}$, we see $x_{\delta_n}\ra p$ but $d_Y(f(x_{\delta_n}),q)\geq\epsilon$, contradicting $\lim_{n\ra\infty} f(x_n)=q$ for every sequence $\set{x_n}\subset E$ such that $x_n\ra p$.
\end{proof}

\ul{Limit properties}: Let $f,g:E\subset(X,d_X)\ra(\mb C,|\cdot|_\mb C)$.
Then if $\lim_{x\ra p} f(x) = A$, $\lim_{x\ra p} g(x) = q$, $p\in E'$,
\begin{enumerate}
	\item $\lim_{x\ra p} (f+g)(x) = A+B$
	\item $\lim_{x\ra p} (f\cdot g)(x) = A\cdot B$ (Inner product if $\mb C$ replaced by $\mb R^k$)
	\item $\lim_{x\ra p} \left(\frac fg\right) = \frac AB$.
\end{enumerate}

\subsection*{Continuity}
\begin{definition}
	Suppose $f:E\subset (X,d_X) \ra (Y,d_Y)$ and $p\in E$.
	We way $f$ is continuous at $p$ if and only if $\forall\epsilon>0,\exists\delta = \delta(\epsilon,p) > 0$ such that $\forall x\in E$ with $d_X(x,p)<\delta \implies d_Y(f(x),f(p))<\epsilon$.
	\ul{Notes}:	
\end{definition}
\begin{itemize}
	\item Continuity is a pointwise property.
	\item $f$ is continuous \ul{on $E$} if $f$ is continuous at every $p\in E$.
	\item $f$ is continuous at an isolated point $p\in E$.
\end{itemize}
\begin{example}
	Fix $\epsilon>0$. Then $d_X(p,p)=0<\delta$ such that $N_\delta(p)\bs\set p=\emptyset \implies d_Y(f(p),f(p))= 0<\epsilon$.
\end{example}

\begin{theorem}
	Suppose $f:(X,d_X)\ra(Y,d_Y)$ is continuous $\iff f\inv(V) = \setc{x\in X}{f(x)\in V}$ is open in $X$ for every open set $V\subset Y$.
\end{theorem}
\begin{proof}
	\say{$\implies$} Suppose $f$ is continuous on $X$. Let $V\subset Y$ be open.
	\ul{\ul{NTS}}: $f\inv(V)$ is open in $X$.
	Let $p\in f\inv(p)$.
	\ul{\ul{NTS}}: $\exists\delta>0$ such that $B_\delta(p)\subset f\inv(V)$.
	Since $p\in f\inv(V)$, $f(p)\in V$.
	So $V$ open $\implies\exists\epsilon>0$ such that $B_\epsilon(f(p))\subset V$.
	Since $f$ is continuous, $\exists\delta=\delta(\epsilon,p)>0$ such that $f(B_\delta(p))\subset B_\epsilon(f(p))\subset V \implies B_\delta(p)\subset f\inv(V) \implies p$ is an interior point of $f\inv(V) \implies f\inv(V)$ is open.

	\say{$\Lla$} next time.
\end{proof}

\end{document}
