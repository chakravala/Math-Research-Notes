\documentclass[]{article}
\usepackage[latin1]{inputenc}
\usepackage{graphicx}
\usepackage[left=1.00in, right=1.00in, top=1.10in, bottom=1.00in]{geometry}

\usepackage{dirtytalk}
\usepackage[normalem]{ulem}
\usepackage{tikz-cd}
\usepackage{units}
\usepackage{algorithm}
\usepackage{algpseudocode}
\usepackage{alltt}
\usepackage{mathrsfs}
\usepackage{amssymb}
\usepackage{amsmath}
\DeclareMathOperator\cis{cis}

% (font shortcuts)
\usepackage{amsfonts}
\newcommand{\mb}[1]{\mathbb{#1}}
\newcommand{\mc}[1]{\mathcal{#1}}
\newcommand{\ms}[1]{\mathscr{#1}}
\newcommand{\mf}[1]{\frak{#1}}

% (arrow shortcuts)
\newcommand{\ra}{\rightarrow}
\newcommand{\lra}{\longrightarrow}
\newcommand{\la}{\leftarrow}
\newcommand{\lla}{\longleftarrow}
\newcommand{\Ra}{\Rightarrow}
\newcommand{\Lra}{\Longrightarrow}
\newcommand{\La}{\Leftarrow}
\newcommand{\Lla}{\Longleftarrow}
\newcommand{\lr}{\leftrightarrow}
\newcommand{\llr}{\longleftrightarrow}
\newcommand{\Lr}{\Leftrightarrow}
\newcommand{\Llr}{\Longleftrightarrow}

% (match parenthesis)
\newcommand{\mlr}[1]{\left|#1\right|}
\newcommand{\plr}[1]{\left(#1\right)}
\newcommand{\blr}[1]{\left[#1\right]}

% (exponent shortcuts)
\newcommand{\inv}{^{-1}}
\newcommand{\nrt}[2]{\sqrt[\leftroot{-2}\uproot{2}#1]{#2}}

% (annotation shortcuts)
\newcommand{\conj}[1]{\overline{#1}}
\newcommand{\ol}[1]{\overline{#1}}
\newcommand{\ul}[1]{\underline{#1}}
\newcommand{\os}[2]{\overset{#1}{#2}}
\newcommand{\us}[2]{\underset{#1}{#2}}
\newcommand{\ob}[2]{\overbrace{#2}^{#1}}
\newcommand{\ub}[2]{\underbrace{#2}_{#1}}
\newcommand{\bs}{\backslash}
\newcommand{\ds}{\displaystyle}

% (set builder)
\newcommand{\set}[1]{\left\{ #1 \right\}}
\newcommand{\setc}[2]{\left\{ #1 : #2 \right\}}
\newcommand{\setm}[2]{\left\{ #1 \, \middle| \, #2 \right\}}

% (group generator)
\newcommand{\gen}[1]{\langle #1 \rangle}

% (functions)
\newcommand{\im}[1]{\text{im}(#1)}
\newcommand{\range}[1]{\text{range}(#1)}
\newcommand{\domain}[1]{\text{domain}(#1)}
\newcommand{\dist}[1]{(#1)}
\newcommand{\sgn}{\text{sgn}}

% (Linear Algebra)
\newcommand{\mat}[1]{\begin{bmatrix}#1\end{bmatrix}}
\newcommand{\pmat}[1]{\begin{pmatrix}#1\end{pmatrix}}
%\newcommand{\dim}[1]{\text{dim}(#1)}
\newcommand{\rnk}[1]{\text{rank}(#1)}
\newcommand{\nul}[1]{\text{nul}(#1)}
\newcommand{\spn}[1]{\text{span}\,#1}
\newcommand{\col}[1]{\text{col}(#1)}
%\newcommand{\ker}[1]{\text{ker}(#1)}
\newcommand{\row}[1]{\text{row}(#1)}
\newcommand{\area}[1]{\text{area}(#1)}
\newcommand{\nullity}[1]{\text{nullity}(#1)}
\newcommand{\proj}[2]{\text{proj}_{#1}\left(#2\right)}
\newcommand{\diam}[1]{\text{diam}\,#1}

% (Vectors common)
\newcommand{\myvec}[1]{\vec{#1}}
\newcommand{\va}{\myvec{a}}
\newcommand{\vb}{\myvec{b}}
\newcommand{\vc}{\myvec{c}}
\newcommand{\vd}{\myvec{d}}
\newcommand{\ve}{\myvec{e}}
\newcommand{\vf}{\myvec{f}}
\newcommand{\vg}{\myvec{g}}
\newcommand{\vh}{\myvec{h}}
\newcommand{\vi}{\myvec{i}}
\newcommand{\vj}{\myvec{j}}
\newcommand{\vk}{\myvec{k}}
\newcommand{\vl}{\myvec{l}}
\newcommand{\vm}{\myvec{m}}
\newcommand{\vn}{\myvec{n}}
\newcommand{\vo}{\myvec{o}}
\newcommand{\vp}{\myvec{p}}
\newcommand{\vq}{\myvec{q}}
\newcommand{\vr}{\myvec{r}}
\newcommand{\vs}{\myvec{s}}
\newcommand{\vt}{\myvec{t}}
\newcommand{\vu}{\myvec{u}}
\newcommand{\vv}{\myvec{v}}
\newcommand{\vw}{\myvec{w}}
\newcommand{\vx}{\myvec{x}}
\newcommand{\vy}{\myvec{y}}
\newcommand{\vz}{\myvec{z}}
\newcommand{\vzero}{\myvec{0}}

%\usepackage[active,tightpage]{preview}
\setlength\PreviewBorder{7.77pt}
\usepackage{varwidth}
\AtBeginDocument{\begin{preview}\begin{varwidth}{\linewidth}}
\AtEndDocument{\end{varwidth}\end{preview}}


\author{Book: Walter Rudin 3rd, Presenter: Maya Chhetri, Notes by Michael Reed}
\title{Mathematical Analysis}
%date{}

\begin{document}
\maketitle

%\begin{abstract}
%\end{abstract}

%\ul{Chapter 3 continued}: 

\begin{recall}
	[Cauchy product] $\sum_{n=0}^\infty a_n$, $\sum_{n=0}^\infty b_n$, then Cauchy product series is:
	$\sum_{n=0}^\infty c_n$, 
	$$c_n = \sum_{k=0}^n a_kb_{n-k} = a_0b_n + a_1b_{n-1} + \dots + a_nb_0.$$
\end{recall}

\begin{theorem}
	\label{thm-3-50}
	Suppose
	\begin{enumerate}
		\item[(a)] $\sum_{n=0}^\infty a_n$ converges absolutely;
		\item[(b)] $\sum_{n=0}^\infty a_n = A$
		\item[(c)] $\sum_{n=0}^\infty b_n = B$
		\item[(d)] $c_n: = \sum_{k=0}^n a_kb_{n-k}$.
	\end{enumerate}
	Then $\sum_{n=0}^\infty c_n$ converges to $AB$.
\end{theorem}
\begin{proof}
	Define sequence of partial sums $A_n:= \sum_{k=0}^n a_k$, $B_n:=\sum_{k=0}^nb_k$, $C_n:=\sum_{k=0}^n c_k$.
	By assumption: $A_n\lra A$ and $B_n\lra B$.
	\ul{Want to show}: $C_n\lra AB$.
	Then 
	\begin{align*}
		C_n&:=\sum_{k=0}^nc_k = c_0+c_1+c_2+\dots+c_n \\
		   &= a_0b_0+(a_0b_1+a_1b_0) + (a_0b_2+a_1b_1+a_2b_0) + \dots + (a_0b_n + a_1b_{n-1}+\dots+a_nb_0) \\
		   &= a_0(\ob{B_n}{b_0 + b_1+\dots+b_n}) + a_1(\ob{B_{n-1}}{b_0+\dots+b_{n-1}}) + a_2(\ob{B_{n-2}}{b_0+\dots+b_{n-2}}) + \dots + a_n\ob{B_0}{b_0} \\
		   &= a_0B_n + a_1b_{n-1} + a_2B_{n-2} + \dots + a_nB_0 
		   = a_0(\beta_n+B) + a_1(\beta_{n-1}+B) + \dots + a_n(\beta_0 + B) \\
		   &= (\ub{A_n}{a_0+a_1+\dots+a_n})B + a_0\beta_n + a_1\beta_{n-1} + \dots + a_n\beta_0 \lra AB
	\end{align*}
	if $a_0\beta_n+a_1\beta_{n-1}+\dots+a_n\beta_0\ra 0$ as $n\ra\infty$, where $\beta_n = B_n-B$ and $B_n=\beta_n+B$.
	
	Let $\gamma_n:= \alpha_0\beta_n + \alpha_1\beta_{n-1} + \dots + a_n\beta_0$.
	\ul{Fix $\epsilon>0$}. \ul{\ul{NTS}}: $\exists N\in\mb N$ such that $|\gamma_n|<\epsilon$ for all $n\geq N$.
	Since $\sum a_n$ converges absolutely, $\alpha:= \sum_{n=0}^\infty |a_n|\in\mb R$.
	Then $|\gamma_n|=|a_0\beta_n + a_1\beta_{n-1} + \dots + a_n\beta_0|$.
	Since $\beta_n\lra 0$ as $n\ra\infty$, $\exists N_1\in\mb N$ such that $|\beta_n|<\epsilon$ for all $n\geq N_1$.
	For $n\geq N_1$,
	\begin{align*}
		|\gamma_n|&=|a_0\beta_n+\dots+a_{n-N_1-1}\beta_{N_1+1} + a_{n-N_1}\beta_{N_1} + \dots + a_n\beta_0| \\
				  &\os{\Delta}{\leq} |a_0\beta_n + \dots + a_{n-N_1-1}\beta_{N_1+1}| + |a_{n-N_1}\beta_{N_1} + \dots + a_n\beta_0| \\
				  &\os{\Delta}{\leq} |a_0|\os{<\epsilon}{|\beta_n|}+|a_1|\os{<\epsilon}{|\beta_{n-1}|}+\dots + |a_{n-N_1-1}|\os{<\epsilon}{|\beta_{N_1+1}|} + |a_{n-N_1}\beta_{N_1} + \dots + a_n\beta_0| \\
				  &< \epsilon \ub{<\alpha}{\sum_{k=0}^{n-N_1-1} |a_k|} + |a_{n-N_1}\beta_{N_1} + \dots + a_n\beta_0| 
				  < \epsilon\alpha + |a_{n-N_1}\beta_{N_1} + \dots + a_n\beta_0|
	\end{align*}
	Keep $N_1$ fixed.
	Letting $n\ra\infty$ and noting that $a_i\ra0$ as $i\ra\infty$, we get $\lim_{n\ra\infty} |\gamma_n|\leq\epsilon\alpha$.
	Since $\alpha>0$ is fixed and $\epsilon>0$ is arbitrary, $\lim_{n\ra\infty} |\gamma_n| = 0$.
\end{proof}

\begin{theorem}
	\label{thm-3-51}
	If $\sum a_n = A$, $\sum b_n = B$, $\sum c_n=C$, where $c_n=\sum_{k=0}^n a_kb_{n-k}$.
	Then $C=AB$.
\end{theorem}
\begin{proof}
	-- In Chapter 8.
\end{proof}

\subsection*{Rearrangement of an infinite series}

$\sum a_n$ is $\sum a_n'$, where $a_n'=a$ where $k_n$ is a bijection from $\mb N$ to $\mb N$.
\begin{example}
	$a_1\ a_2\ a_3\ a_4\ a_5\ a_6\ a_7\ \cdots$, 
	
	$a_1\ a_3\ a_2\ a_5\ a_7\ a_4\ \cdots$ (rearrangement),
	
	$a_1\ a_3\ a_5\ a_7\ \cdots\ a_{2n+1}\ \cdots\ a_2\ a_4\ a_6\ a_8\ \cdots$ (not a rearrangement).
\end{example}
\begin{example}
	Alternating harmonic series: $\sum_{n=1}^\infty (-1)^{n-1}\frac1n = 1-\frac12+\frac13-\frac14+\frac15-\dots = \ln 2$.
	
	Rearrange: 2 odd terms followed by even term $1+\frac13-\frac12+\frac15+\frac17-\frac14+\frac19+\frac1{11}-\frac16+\dots$.
	
	\ul{\ul{Show}}: Sum of rearranged series is $\frac32\ln2$.
	$$\ln 2 = 1-\frac12+\frac13-\frac14+\frac15-\frac16+\frac17-\frac18+\dots.$$
	Multiply by $\frac12$:
	$$\frac{\ln2}{2} = 0+\frac12+0-\frac14+0+\frac16+0+\dots.$$
	Add the two above:
	$$ \frac32\ln2 = 1+0+\frac13-\frac12 + \frac15 + \frac17-\frac14+\dots,$$
	so rearranged series sum is $\frac32\ln 2$.
\end{example}

\begin{theorem}
	Suppose $\sum a_n$ converge absolutely.
	Then \ul{every} rearrangement $\sum a_n'$ of $\sum a_n$ converges, and they all converge to the same sum.
\end{theorem}
\begin{proof}
	Let $\sum a_n'$ be a rearrangement of $\sum a_n$.
	Let $s_n$ and $s_n'$ be sequences of partial sums of $\sum a_n$ and $\sum a_n'$ respectively.
	Also $s_n\lra s$ (say). \ul{\ul{NTS}}: $s_n'\ra s$.
	Let $\epsilon>0$ be fixed.
	\ul{NTS}: $\exists N\in\mb N$ such that $|s_n'-s|<\epsilon$ for all $n\geq N$.
	Now, 
	\begin{align*}
		|s_n'-s|&=|s_n'-s_n+s_n-s| \\
				&\leq |s_n'-s_n|+|s_n-s|
	\end{align*}
	Since $\sum a_n$ converges absolutely, $\exists N_2\in\mb N$ such that $m\geq n>N_2$
	$$\sum_{i=n}^m |a_i| < \frac\epsilon2.$$
	Choose $p\in\mb N$ such that $\set{1,2,\dots,N_2}\subset\set{k_1,\dots,k_p}$.
	Then for $n>p$
	\begin{align*}
		s_n'-s_n| &= |(a_1'+a_2'+\dots+a_n')-(a_1+a_2+\dots+a_n)| &\qquad \text{-- \ul{cancel terms}} \\
				  &\leq \sum_{i>n+1}^{k_p} |a_i| < \frac\epsilon2
	\end{align*}
\end{proof}

\end{document}
